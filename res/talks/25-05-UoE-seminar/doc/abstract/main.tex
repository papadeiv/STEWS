\PassOptionsToPackage{dvipsnames, table}{xcolor}
\documentclass[dvipsnames, table, 11pt]{article}
\pagenumbering{gobble}

\textwidth=7in
\textheight=9.5in
\hoffset=-1in
\voffset=-1in
\parskip=10pt
\parindent=0pt

\usepackage{style}

\begin{document}
\title{Hints of collapse: tipping points and warning signals}

\author[*]{\textbf{PhD student:} Davide Papapicco}
\author[*]{\authorcr \textbf{Supervision:} Graham Donovan}
\author[*]{Lauren Smith}
\author[$\dagger$]{Merryn Tawhai}

\affil[*]{Department of Mathematics, Faculty of Science, University of Auckland, Auckland, New Zealand}
\affil[$\dagger$]{Auckland Bioengineering Institute, University of Auckland, Auckland, New Zealand}
 
\date{}

\maketitle

\begin{abstract}
        Sometimes some natural and social systems exhibit abrupt, sudden and often unforeseen changes with significant, long-lasting effects.
        These rapid changes can be thought as fingerprints of instability: critical transitions connecting two regimes at equilibrium.
        It then appears that these critical transitions occur when a system crosses some invisible critical threshold causing the sudden change.
        In turn we can imagine these invisible thresholds to be tipping points that the system approaches at equilibrium in a slow, seemingly unperturbed fashion and, as it crosses them, it undergoes a dramatic fast departure to a new state. 

        The disappearance of species from an ecosystem (Figure \ref{fig1}), stock market crashes (Figure \ref{fig2}) and extreme paleoclimate events (Figure \ref{fig3}) are all hallmarks of this dynamical interpretation of crises in nature and society.
        Unsurprisingly several efforts in the past two decades have tried to decode the underlying structures of these phenomena.
        Particular emphasis has been put in the derivation of warning signs that precede the collapse, mainly motivated by the fact that, following a critical transition, it could be impossible for a system to recover the previous stable state.

        \begin{figure}[H]
            \begin{subfigure}[b]{0.325\textwidth}
                    \centering 
                    \includegraphics[keepaspectratio, width = \linewidth]{../../fig/abstract/fig1.png}
                    \caption{}
                    \label{fig1}
            \end{subfigure}
            \hfill
            \begin{subfigure}[b]{0.325\textwidth}
                    \centering 
                    \includegraphics[keepaspectratio, width = \linewidth]{../../fig/abstract/fig2.png}
                    \caption{}
                    \label{fig2}
            \end{subfigure}
            \hfill
            \begin{subfigure}[b]{0.325\textwidth}
                    \centering 
                    \includegraphics[keepaspectratio, width = \linewidth]{../../fig/abstract/fig3.png}
                    \caption{}
                    \label{fig3}
            \end{subfigure}
            \caption{Examples of natural and human critical transitions.}
        \end{figure}

        Mathematical modelling of a system displaying tipping points often comes in the form of a low-dimensional, non-autonomous, stochastic differential equation (SDEs).
        Early warning signals are then obtained as (often statistical) quantities that can be computed from the timeseries of such dynamical systems.
        While bifurcation analysis and numerical continuation form the cornerstone of the study of some types of these tipping, much is still left unclear regarding the mechanisms driving the collapse in different regimes and higher dimensions.
        In applications, a particular concern seems to be the universality and robustness of a specific early warning when applied to real-world data rather than the synthetic timeseries of a stochastic process.
        Several limitations concerning the uncertainty of the model generating the data, the non-stationarity of the process (and its associated Fokker-Planck equation) and inherent irregularities in the observations significantly diminish the predictive power of current warning signals.

        We are developing a novel numerical method providing probability estimates of noise-induced tipping (N-tipping).
        The probability of collapse of an observed realisation is inferred by association with an overdamped particle in a potential well subject to stochastic fluctuations.
        Reconstruction of the potential landscape from distribution data is obtained via linear-least squares regression and statistical error analysis shows $L_{1}-$ convergence.
        Applications to $1-$dimensional SDEs are discussed and future generalisation to high-dimensional (spatially extended) dynamical systems will be considered.
\end{abstract}
\end{document}
