\PassOptionsToPackage{dvipsnames, table}{xcolor}
\documentclass[dvipsnames, table, 11pt]{article}

\textwidth=7in
\textheight=9.5in
\hoffset=-1in
\voffset=-1in
\parskip=10pt
\parindent=0pt

\usepackage{style}

\begin{document}
\title{Slowly, then all at once: Uncovering the dynamics of a catastrophe}

\author[*]{\textbf{PhD student:} Davide Papapicco}
\author[*]{\authorcr \textbf{Supervision:} Graham Donovan}
\author[*]{Lauren Smith}
\author[$\dagger$]{Merryn Tawhai}

\affil[*]{Department of Mathematics, Faculty of Science, University of Auckland, Auckland, New Zealand}
\affil[$\dagger$]{Auckland Bioengineering Institute, University of Auckland, Auckland, New Zealand}

\date{}

\maketitle

\begin{abstract}
        Many natural and human complex systems evolve on a slow timescale and are stable with respect to external perturbations. 
        However, these systems can experience sudden rapid departures from their natural equilibrium, known as tipping events, which often bring catastrophic and unrecoverable repercussions.
        Extreme paleoclimate events, ecosystems’ collapse and economic crises are some examples of dynamical systems evolving slowly around an equilibrium until a tipping point causes a fast, unforeseen critical transition outside the basin of attraction and onto a new, unhealthy state. 
        An early-warning signal is a (often statistical) quantity that provides forewarning of a major tipping point based on the structure of the timeseries data while in hyperbolic regime.
        Precursing phenomena such as critical slowing down or flickering between alternative states in metastable regimes are easily captured by an increase in variance or skewness respectively. 
        However they are limited to low-dimensional dynamical systems that do not include spatial information and pattern-forming instabilities.
        In this poster we will review the fundamental issues plaguing the application of early-warning signals theory for spatially-extended stochastic PDEs and introduce a novel, prototypical approach based on the escape time from a potential well from a probabilistic perspective.
\end{abstract}
\end{document}
