\documentclass[../main.tex]{subfiles}
\begin{document}
\subsection{A comprehensive catalogue of spatio-temporal early-warning signals}\label{subsec3.3}
The two indicators of pattern-forming tipping events discussed above are but among the latest entries of a much larger ecosystem of EWS that have been proposed throughout the past two decades. 
Their distinctive property w.r.t. the rest of the earlier measures is that they both combine spatial and temporal information rather than looking at either dimensions individually.
As addressed in the earlier literature review, the discovery and characterisation of tipping events was preceeded by the observation of critical phenomena s.a. CSD and flickering whose measurement through statistical and dynamical indicators become later EWS.
As discussed in the previous Chapter, the robustness of the signals of these precursors is of uttermost importance.
Specifically an optimal EWS of tipping events should address: 
\begin{itemize}
     \item the existence (or lack thereof) of a sound mathematical framework linking the properties and structure of the models exhibiting a tipping event with the underlying theorised or observed phenomena (i.e. CSD, flickering etc...);
     \item a reasonably wide scope of applicability (or degree of universality) to different natural systems modelled by the same mechanism (e.g. reaction-diffusion equations);
     \item the sensitivity of the indicator to downsampling and other practical limitations in the monitoring of complex real-world systems (s.a. climate change, ecosystem collapse, onset of asthma etc...);
     \item the reliance of the methods detecting the signal on single and multivariate timeseries analysis, including detrending and other preprocessing techniques.
\end{itemize}
Addressing how robust a spatio-temporal EWS is w.r.t. the aforementioned metrics is arguably among the most important tasks ahead in the discovery and characterisation of catastrophic regime shifts. 
To conclude this Chapter, here we limit ourselves in categorizing the indicators of the precursors discussed in the literature review.
A tabulated list of indicators is not new and as a matter of fact we found $5$ works \cite{Lenton11, Dakos12a, Scheffer12, Kefi14, Scheffer15} published in the first half of the 2010s compiling a comprehensive collection of the EWS being proposed until then and their connections to the relevant precursors.
Remarkably \cite{Scheffer15} was also accompained by the creation of a website\footnote{\url{https://www.early-warning-signals.org/}} collecting and detailing all the progresses that have been done insofar in the detection of both ad-hoc and generic EWS.
In Table \ref{tab2.1} below we merge these lists by selecting only those indicators that have shown promises of robustness to some degree while enriching them with newer signals proposed after the publication of the aforementioned papers.
We remark that the discovery of critical phenomena preceeding a critical transition and characterising them using indicators has been an ongoning effort by multiple communities with a strong focus in the ecological and climate sciences.
This means that the mathematical ground upon which many of those precursors are based varies greatly, from semi-rigorous and well-developed frameworks for idealised cases \cite{Kuehn11, Kuehn13} to more pragmatic, data-driven and intuition-based indicators of experimental monitorings for scientific applications \cite{Scheffer09,Lenton11}.
\begin{table}[H]
\centering
\begin{tabular}{|lll|}
 \hline
 Indicator & Phenomenon & First proposed \\
 \hline
 recovery time & CSD & 1995 (\cite{Ives95}) \\
 \rowcolor{lightgray} spectral reddening & CSD & 2003 (\cite{Kleinen03}) \\
 autoregressive model fitting & CSD & 2004 (\cite{Held04}) \\
 \rowcolor{lightgray} lag-1 autocorrelation & CSD & 2004 (\cite{Held04}) \\
 \textbf{size distribution of patches} & Turing instability & 2004 (\cite{Rietkerk04}) \\
 \rowcolor{lightgray} \textbf{spatial variance} & CSD, flickering, Turing instability & 2005 (\cite{Oborny05}) \\
 \rowcolor{lightgray} variance & CSD, flickering & 2006 (\cite{Carpenter06}) \\
 detrended fluctuation analysis & CSD & 2007 (\cite{Livina07}) \\
 \rowcolor{lightgray} \textbf{leading eigenvalue of the covariance} & CSD, Turing instability & 2008 (\cite{Carpenter08}) \\
 skewness & flickering & 2008 (\cite{Guttal08}) \\
 mean exit time & CSD & 2008 (\cite{Guttal08}) \\
 kurtosis & flickering & 2009 (\cite{Biggs09}) \\
 \textbf{spatial skewness} & flickering, Turing instability & 2009 (\cite{Guttal09}) \\
 \rowcolor{lightgray} \textbf{spatial spectral reddening} & CSD, Turing instability & 2010 (\cite{Carpenter10}) \\
 \rowcolor{lightgray} \textbf{spatial correlation} & CSD, \newline Turing instability & 2010 (\cite{Drake10}) \\
 potential analysis & flickering & 2010 (\cite{Livina10}) \\
 \textbf{traveling wavespeed} & flickering & 2013 (\cite{Kuehn13}) \\
 \rowcolor{lightgray} \textbf{leading mode of the DMD} & CSD, Turing instability & 2020 (\cite{Gottwald20}) \\
 \textbf{mutual information} & CSD & 2024 (\cite{Deb24}) \\
 \hline
 \end{tabular}
 \caption{Catalogue of robust EWS of critical transitions sorted from top to bottom by the year in which they were first proposed over the past $2$ decades. In bold we indicate spatial and spatio-temporal indicators. 
 In gray we highlight the most promising EWS in spatially extended dynamical systems which will also be the focus of the next Chapters.}
 \label{tab2.1}
 \end{table}
\end{document}
