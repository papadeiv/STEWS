\documentclass[../main.tex]{subfiles}
\begin{document}
\subsubsection{Inversion of the equilibrium distribution}\label{subsubsec4.2.1}
Solutions of the stationary FPE \cite[Theorem 3.1, p. 3]{StocProc} can be used to reconstruct the potential function.
Assuming the appropriate BCs for the FPE, we can write the equilibrium distribution as
\begin{equation}\label{eq4.2.1.1}
     p(x) = N\,e^{-\frac{V(x)}{D}}\,,
\end{equation}
where $V(x)$ is the potential function of the drift term of the SDE and $D=\frac{\sigma^{2}}{2}$ is the diffusion coefficient.
A simple inversion gives us
\begin{equation}\label{eq4.2.1.2}
     V(x) = -D\ln\bigg(\frac{p(x)}{N}\bigg) = D\ln N - D\ln p(x)\,,
\end{equation}
which tells us that if we have access to the (approximate) equilibrium distribution $p(x)$ and we know the level of noise in the system $D$ then we can recover the shape of the potential function $V(x)$ up to an arbitrary shift given by $D\ln N$, where $N>0$ is a normalisation constant for $p$. 
\begin{example}[label=ex4.2.1.1]{}{}
     Let us consider a saddle-node normal form for the drift of a SDE with additive noise
     \begin{equation*}
          dx = -(\mu+x^{2})dt + \sigma dW\,.
     \end{equation*}
     The stationary FPE equation reads
     \begin{equation*}
          \newprime{\Big(\newprime{p}+\frac{1}{D}(\mu+x^{2})p\Big)} = 0\,.
     \end{equation*}
     We follow the procedure outlined in proof of \cite[Theorem 3.2, pp. 4-6]{StocProc} to solve the second-order IVP and specify that $p(x), \newprime{p}(x)\to0$ as $x\to+\infty$.
     Integrating the above in $[x,+\infty)$ yields
     \begin{equation*}
          \newprime{p}(x)+\frac{\mu+x^{2}}{D}p(x) = 0\,.
     \end{equation*}
     The integrating factor is 
     \begin{equation*}
          h(x) = e^{\frac{1}{D}\int_{}^{}(\mu+x^{2})dx} = e^{\frac{\mu x + \frac{x^{3}}{3}}{D}}\,,
     \end{equation*}
     which, when multiplied to both sides of the first-order, homogeneous ODE, gives us
     \begin{equation*}
          \newprime{\Big(e^{\frac{\mu x + \frac{x^{3}}{3}}{D}}p(x)\Big)} = 0\;\Rightarrow\;p(x) = N\,e^{-\frac{\mu x + \frac{x^{3}}{3}}{D}}\,,
     \end{equation*}
     where $N>0$ is a normalisation constant.
     By definition
     \begin{equation*}
          \int_{-\infty}^{+\infty}p(x)\,dx = 1 = N\,\int_{-\infty}^{+\infty}e^{-\frac{\mu x + \frac{x^{3}}{3}}{D}}dx\;\Rightarrow\;N=\frac{1}{\int_{-\infty}^{+\infty}e^{-\frac{\mu x + \frac{x^{3}}{3}}{D}}dx}\,,
     \end{equation*}
\end{example}
\begin{example_continued}
     which gives us
     \begin{equation*}
         p(x) = \bigg(\int_{-\infty}^{+\infty}e^{-\frac{\mu x + \frac{x^{3}}{3}}{D}}dx\bigg)^{-1}e^{-\frac{\mu x + \frac{x^{3}}{3}}{D}}\,. 
     \end{equation*}
     Notice that since $e^{-\frac{\mu x + \frac{x^{3}}{3}}{D}}\to+\infty$ as $x\to-\infty$ then $N=\infty$ and as such we conclude that the stationary solution of the FPE for the saddle-node normal form does not exist over the domain $(-\infty,+\infty)$.
     In order to get a normalisable solution that exists we need to truncate the domain to $[a,+\infty)$.
     When we do that we impose reflecting (i.e. zero-flux) BCs s.t. $\newprime{p}(a)=0,p(a)\neq0$.
     We choose the lower boundary of the domain to be the unstable equilibrium of the saddle-node normal form, i.e. $a = -\sqrt{-\mu}=:x_{u}$, which obviously entails that $\newprime{V}(x_{u}) = 0$ and $\pprime{V}(x_{u})<0$.
     We thus retrieve a stationary solution of the form
     \begin{equation*}
         p(x) = \bigg(\int_{x_{u}}^{+\infty}e^{-\frac{\mu x + \frac{x^{3}}{3}}{D}}dx\bigg)^{-1}e^{-\frac{\mu x + \frac{x^{3}}{3}}{D}}\,. 
     \end{equation*}
\end{example_continued}
% The problem with the normalisation constant
\subfile{par4.2.1.1}
\end{document}
